\documentclass[../main.tex]{subfiles}
\begin{document}
  \section{Lambda演算}
  \indent 在上一章中,我们已经学会了如何形式化地描述一门语言。
  在本章中,我们将介绍一门编程语言:$\lambda$演算。
  在后续的内容中,我们都将基于$\lambda$演算来进行讲解。

  \subsection{XXX是世界上最好的语言!}
    \indent 如何让办公室里的程序员们吵起来?
    答案是在办公室里大声喊道:“某某语言是世界上最好的语言!”。
    \footnote[1]{这个笑话的版本包括但不限于Java,Python或者PHP}
    这足以证明了编程语言的百花齐放,以及每一个程序员心中都有自己的哈姆雷特。
    既然这本小册子所需要介绍的类型系统是基于编程语言的,那么“用什么语言来描述类型系统”
    就是一个无法避开的话题。

    \indent 你也许是一个传统的C语言程序员,也有可能是熟练的Java使用者。
    如果要跟随潮流的话,说不定你也可以是一个使用Python的AI训练师。
    不过非常抱歉的是:无论你在学习或者工作中使用怎么样的语言,我们都不会用它来讨论类型系统。
    归根结底,这些语言都过于“复杂”了:为了能支持工业软件的开发,这些语言拥有大量的特性,
    而这些特性很有可能会干扰我们对于类型系统的研究。

    \indent 为此,我们要在本章提出一个更为精简的“语言”。
    它需要简单易用,又要有能力完成其他编程语言所能完成的任务。
    这个“语言”就是$\lambda$演算。
    我们在这里给“语言”二字加上了引号,这是因为$\lambda$演算本身其实是一个形式系统:
    在计算机出现之前,$\lambda$演算就已经登上了历史舞台。
    \footnote[2]{它由数学家阿隆佐·邱奇在20世纪30年代首次发表。\cite{lamwiki}}
    尽管如此,$\lambda$演算有能力模拟一个最基本的编程语言。\cite{lamwiki}
    
    \indent 在本章剩下的内容中,我们将讨论$\lambda$演算的基本语法和语义。
    为了不一次性灌输太多的内容,我们将从无类型的$\lambda$演算开始。
    在理解了$\lambda$演算的基本概念后,我们再慢慢地将类型系统加上。

  \subsection{无类型的Lambda演算}

  \subsection{究竟何为类型?}

  \subsection{有类型的Lambda演算}

  \subsection{我们是对的吗?}

  \begin{thebibliography}{1}
    \bibitem{lamwiki} Zh.wikipedia.org. 2022. λ演算 - 维基百科,自由的百科全书. [online] Available at: <\url{https://zh.wikipedia.org/wiki/%CE%9B%E6%BC%94%E7%AE%97}> [Accessed 3 November 2022]. 
  \end{thebibliography}

\end{document}