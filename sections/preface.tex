\documentclass[../main.tex]{subfiles}
\begin{document}
  \section*{前言}
  \indent 在阅读完著名的TAPL(《Types and Programming Languages》)后,笔者决定做一些什么。这一“决定”就是这本《Type Typer Yet Typer》。
  \newline
  \indent TAPL固然是一本好书,作者Benjamin C. Pierce详细介绍了类型系统的理论知识,并辅佐以大量的实例。这本书也将伴随笔者后续的学习和工作。
  既然如此,为什么不索性继续使用TAPL呢?
  笔者曾经向一些朋友展示过笔者的工作内容:大部分人在看到满篇的公式证明后惊呼“太恐怖了!”(包括笔者自己第一次见到这本书的时候)。
  这样的反应大概算不上对这本书的赞赏,仿佛提起FC你的第一反应是魔界村而不是马里奥,提起DOTA2只记得钢琴手卡尔刷新12连。
  \newline
  \indent 因此,笔者希望能够拥有一个更简单的版本——简单到只要学过编程就能轻松理解其中的含义。
  很多人都喜欢人工智能、深度学习,笔者认为这很大程度上是众多开源框架(比如PyTorch)和模型(真不熟,不列举了)的功劳:
  这些代码和模型大大降低了学习的门槛,只要你会写Python(尽管笔者并不喜欢Python),你也可以拥有一个炫酷的手写数字识别。
  在这个层面上,“降低门槛”是一件多多益善的事。
  \newline
  \indent 笔者选择在刚刚读完TAPL,就来尝试制作这本《Type Typer Yet Typer》。这意味着笔者现在自己也是半瓶醋的状态(对知识的了解和写作水平都是)。
  也许有人(而且是不少人)会认为“等你深谙这个领域的各种细枝末节了,再来写也不迟”,而笔者认为这个时期是避开“知识的诅咒”的最佳时间
  (一种认知偏差,指人在与他人交流的时候,下意识地假设对方拥有理解所需要的背景知识。这段引用自维基百科,但我并不想在前言部分就开始列参考文献)。
  因此,笔者认为这更像是一本“学习笔记”,而不是专业权威的参考书目,管《Type Typer Yet Typer》叫做“小册子”也更加合适。
  这本小册子将会随着笔者能力的精进而不断完善,笔者也希望这本书可以对你有所帮助。
\end{document}
