\documentclass[../main.tex]{subfiles}
\begin{document}
  \section{如何描述一门编程语言?}
  \indent 到目前为止,我们都是在以一种非形式化的方式来描述某一门编程语言,
  就像我们在大部分编程书籍上可以看到的那样:

\begin{lstlisting}[language=c]
if (condition) {
  // do something
}
else if (another condition) {
  // do something
}
else {
  // do something
}
\end{lstlisting}

  \indent 这固然很好!非形式化的描述直观易懂。
  非常可惜的是:非形式化的描述在我们后续的内容中并不好使。
  为了严格证明我们所构造的类型系统的“正确性”(何为“正确的”类型系统,我们会在之后的章节中解释),
  我们需要使用形式化的方式来描述编程语言,即:用更“数学”的方式表示“这段代码要做什么”。

  \indent 尽管这本小册子将“避重就轻”,为大家省去大量的证明工作,
  了解这些形式化的语言仍然很有必要!
  这些知识可以帮助你理解你所学习和使用的编程语言:例如语言的特性和用法,
  编译器/解释器/运行时环境的原理,等等。\cite{edu}
  因此,我们将花一整章来讨论编程语言的形式化描述。
  为了让这些理论知识更加容易被接受,我们将从小学的语文课开始。

  \subsection{语言的语法}
  \indent 想象一下你回到了你的童年!你正坐在小学的教室里,
  老师正在教你理解一个很长很长的句子。
  在理解句子之前,老师一定会先教给你三样东西:主语、谓语和宾语。
  
  \indent 没错!对于一门自然语言来说,一个简单的句子通常由这三个元素构成。
  无论是汉语、英语或者其他语言:

  \begin{itemize}
    \item 我(主语)打断了(谓语)他的动作(宾语)。
    \item It(主语) is(谓语) a disaster(宾语)!
    \item ……
  \end{itemize}

  \indent 在语文老师(也许还有英语老师)的帮助下,
  我们得到了一个结论:一个句子可以通过主语谓语宾语的拼接组成。
  这就是一门语言的语法(grammar)。
  而上面的“主谓宾结构”构成了一条语法规则:

  $$Sentence = (Subject) (Verb) (Object)$$
  
  \indent 编程语言作为另一类“语言”(另类在我们用它和计算机沟通而不是和人沟通),
  是否可以用类似的方法来理解呢?答案是肯定的。
  在计算机科学中,编程语言的语法(syntax)\footnote[1]{注:这里的“语法”与自然语言的语法仍有很大的差别。自然语言的语法包括句法、词汇形态学和音系学\cite{grammar},而编程语言中的语法其实仅指句法。因此这里的英文是syntax而不是grammar。}
  包含一系列的语法规则,
  这些规则规定了符号如何进行排列,从而得到有意义的计算机程序语句或表达式。\cite{syntaxwiki}

  \subsection{如果只有语法的话……}
  % TODO

  \subsection{操作语义(Operational Semantics)}
  % TODO

  \begin{thebibliography}{3}
    \bibitem{edu} 知乎专栏. 2018. 浅谈国内高校编程语言教育. [online] Available at: <\url{https://zhuanlan.zhihu.com/p/43914842}> [Accessed 2 October 2022].
    \bibitem{syntaxwiki} En.wikipedia.org. 2022. Syntax (programming languages) - Wikipedia. [online] Available at: <\url{https://en.wikipedia.org/wiki/Syntax_(programming_languages)}> [Accessed 2 October 2022].
    \bibitem{grammar} 王, 力., 2015. 中国语法理论. 中华书局, p.6.
  \end{thebibliography}
\end{document}