\documentclass[../main.tex]{subfiles}
\begin{document}
  \section{类型与编程语言}
  \subsection{软件工程师们是如何解决问题的}

  \indent 欢迎来到软件工程的世界!在这里,请允许我假定你已经掌握了一些基本的编程技术(如果还没有的话,为什么不现在就开始学习呢?),
  例如C,C\#,或者Python编程。无论你掌握了哪一门语言,你都已经具备了编写程序的能力,可以开发属于自己的软件了!让我在这里恭喜你!

  \indent 等等!你我其实都心知肚明:这仅仅只是一个开始(如果你确实是一个新晋的程序员的话)。
  你的程序也许并不像你想象中那么完美,比如下面的这个小小的计算程序:

\begin{lstlisting}[language=python]
area = 45
width = input("please input the width of rectangle:")
print(area / int(width))
\end{lstlisting}

  \indent 这是一个非常简单的程序,仅仅只包含3行代码。它可以帮助你计算“当一个矩形的面积为45,宽为width时,矩形的高是多少”。
  但是这样的程序存在着非常严重的缺陷:当用户输入0时,程序将会崩溃(0不能作为除数!):

\begin{lstlisting}
Traceback (most recent call last):
File "script.py", line 5, in <module>
  print(area / int(width))
ZeroDivisionError: division by zero

Exited with error status 1
\end{lstlisting}

  \indent 甚至更糟糕!用户可能根本不会输入一个数字(也许不小心输入了家庭住址)!
  这些情况都会导致我们的程序崩溃,后续的逻辑根本不会执行,从而让用户为此抓狂!
  事实上,不仅仅是我们的程序会遇到这样的问题。很多知名的大型软件(比如Windows系统)的开发者,
  他们也需要不停地思考:怎么样才能让软件更加“好用”呢?

  \indent 这里的“好用”一词实属模糊,不同的人对于“好用”的定义是完全不一样的。
  不过,我们的软件还是应该至少保证两点:按照我们的预期产生输出,并且程序不要崩溃。
  如果程序的输出是错的(比如告诉你$16 \times 55 = 28$),我们的用户压根没法指望着靠这个程序完成什么工作;
  如果程序在运行过程中彻底崩溃了,用户正在进行的工作将付之东流(应该不会有人在Word文档崩溃的时候才发现自己写了两小时的报告忘记保存了吧?)。

  \indent 为此,软件工程师们(以及众多学术界的大牛们)想尽一切办法,提出了一系列的措施,
  目的就是为了减少程序员在开发过程中犯错的次数,从而使用户最终拿到的软件更加可靠。
  这一系列的措施可以大致归纳为三类:\cite{sf}
  \begin{itemize}
    \item 错误终归是由于人引发的。如果不用熬夜甚至通宵写代码的话,代码的正确性就不会依赖于血液中咖啡因的浓度了吧。因此,合理安排项目进度,积极与用户沟通等方法都有助于减少软件中错误的产生。这样的方法我们称之为项目管理;
    \item 如果老的代码是可靠的,那么只需要“复用”以前的工作就好了吧!工程师们通常会设计出一些“模式”,通过使用这些模式来组织代码,后人就能方便地复用这些成果,从而减少新产生的错误的数量。
    \item 会产生错误就是由于程序员写代码的时候太懒散了!多给他们立一些规矩吧!比如“这个地方不允许这样写”,“这个地方你必须输出这个”……这些规矩最后会被反应到编程语言、测试框架或者验证程序上。
  \end{itemize}

  \indent 限于我们的精力和时间(以及篇幅),如此之多的手段我们将只关注编程语言本身。
  特别地,我们将关注编程语言中不同的“类型”,以及如何根据这些类型信息确保程序的正确性。

  \subsection{编程语言中的类型}
  \indent “物以类聚,人以群分”(出自《战国策·齐策三》)。分类在不同领域不同学科都有着不俗的影响力。在计算机和软件工程中亦是如此。
  然而,在“什么是类型”这个问题上,大家很难达成一致。
  
  \indent 从历史角度上来说,C语言引入“类型”是为了区分浮点与整数运算\cite{typeytb}。
  底层硬件中,浮点和整型的存储和运算存在很大的不同。因此,同样是两个数字相加,整型和浮点型需要使用不同的操作数来表示。
  你也可以说:类型是为了区分不同的底层内存表示。

  \indent 从更抽象的角度来说,类型是为了区分不同的变量和值:
  某些值可以被赋给这个变量,而某些值却不可以;同一类型的值可以执行一些相同的操作。
  例如下面的这段C++程序:

\begin{lstlisting}[language=c++]
struct Foo { int a; };
struct Bar { float a; };

Foo foo;
Bar bar;
\end{lstlisting}

  \indent 在这段程序中,Foo和Bar分属两个不同的类型。
  你无法将一个Foo类型的值赋给bar,也无法将一个Bar类型的值赋给foo。
  对于foo变量(或者其他Foo类型的值),你可以访问到一个类型为整型的字段a;
  对于bar变量(或者其他Bar类型的值),你可以访问到一个类型为浮点型的字段a。
  对于大部分程序员来说,这也是最直观最容易理解的“类型”。

  \indent 很遗憾,你我所熟悉的这些“类型”定义并不足以支撑起这个领域的理论研究。
  为了能够形式化地证明程序的正确性,我们需要更加抽象——或者说,更有“数学味”的类型定义。
  尽管如此,我们并不打算在此展开讨论这个更数学化的定义:过早地接受这些概念并无益处。
  因此,请允许我在此卖一个关子。\footnote[1]{我们将在第?章再来讨论这个问题。}
  
  \indent 作为补偿,在本章的最后,我们将讨论一个轻松的话题:
  不同的编程语言之间,类型系统有何不同?

  \subsection{类型系统的分类}
  \indent 不同的编程语言,其所拥有的类型系统也会大不相同。
  例如在一些更加高级的语言(比如Java)中,你找不到“指针”这一概念;
  某些语言也会简单地将整数与浮点数合称为number类型(比如TypeScript)。

  \indent 这样讨论下去就没完没了了!因此,我们不再考虑这些细枝末节的东西,
  转而讨论一些更为“通用”的特性:

  \begin{itemize}
    \item 类型检查发生在编译期还是运行期?
    \item 类型系统允不允许隐式转换?
    \item 类型系统如何判断两个类型是否相同?
    \item 类型系统支不支持类型推断?
    \item ……
  \end{itemize}

  \indent 我们来逐一讨论以上提出的问题。

  \subsubsection*{静态类型与动态类型}
  \indent 在上一节中我们看到的C语言,就是一个典型的静态类型的例子。“静态”一词意味着检查是在编译期执行的。
  例如这样的一段代码:

\begin{lstlisting}[language=c]
float a = "abc";
\end{lstlisting}

  \indent 编译器会友好地提醒你:你在试图用一个字符数组初始化一个浮点数,而这两个类型是完全不兼容的!
  你的程序将无法被编译,也就不会有可执行文件(或者链接库等其他二进制产物)生成。

  \indent 与之相对的则是动态类型。“动态”一词意为着对类型的检查要等到运行时才会被执行。
  这样的类型系统会对运行时的代码产生一定的性能负担(除了执行用户的代码外,还要随时检查类型),
  但其相较于静态类型的语言有着更高的灵活度,因此被大量用于脚本语言中,比如Python:

\begin{lstlisting}[language=python]
print("Hello, World!" & 2)
\end{lstlisting}

  运行这段代码,解释器会告诉你:无法将字符串和整数类型这样运算。而你的代码的执行之旅也到此结束。
  不过,如果这段代码没有被执行到,解释器是不会有任何的意见的:

\begin{lstlisting}[language=python]
if False:
	print("Hello, World!" & 2);
else:
	print("Yes!")
\end{lstlisting}

  \indent 除此之外,还有一些语言(比如C\#)秉持着中庸的思想,认为同时支持静态类型检查和动态类型检查是一个不错的选择。
  这样的类型系统被称为渐进类型(Gradual Typing)。 \cite{gradualwiki} 在C\#中,绝大多数的类型依然是静态的。
  但是其提供了一个dynamic关键字。使用dynamic修饰的变量将会被视作动态类型。

  \subsubsection*{强类型与弱类型}
  \indent 尽管大部分语言都对整数和浮点数做了区分(int和float),但是大家在进行计算的时候似乎并没有做过多的区分:
  一个整数可以简单地与一个浮点数(甚至是双精度的浮点数)进行四则运算,从而得到另一个浮点数。
  这明明是两种不同的类型!我们的类型系统坏掉了吗?

  \indent 事实上,这是一种名为“隐式转换”(Implicit Conversion)的机制在起作用。类型系统在看到一个整数和浮点数同时运算后,
  做出了“将整数转化成浮点数然后再试试吧”的决策,而这个决策并没有直接告知程序员。
  一个类型系统的强弱程度,取决于有多少类型可以在这样的“潜规则”下进行相互转换。

  \indent 隐式转换很方便,但是也会带来一系列的问题。C语言就是一个非常典型的弱类型语言,其允许一个整数被隐式转换到布尔值上,具体的转换规则为:
  \begin{itemize}
    \item 如果这个整数为0,那么转换结果为false
    \item 否则,转换结果为true
  \end{itemize}

  \indent 在这样的规则下,隐式转换为下面的代码埋下隐患:

\begin{lstlisting}[language=c]
int i = 0;
int j = 10;
while (i = j) {
  // do something
  ++i;
}
\end{lstlisting}

  \indent 可怜的程序员!在经历了通宵的加班后,他并没有发现他将$==$写成了$=$。
  而编译器会自动将$i = j$返回的整数转化为布尔值,从而让这段错误的代码通过了编译!
  用户在拿到程序后,就会发现程序卡死在了某个地方(就是这个死循环)。
  如果是一门强类型的语言,编译器(或者解释器)也许就会提示:不能将整数视作布尔值!
  这个问题就能迎刃而解。

  \indent 当然了,类型过“强”也不见得是一件好事。在OCaml语言中,浮点运算有“专门”的运算符。
  这意味着在我们的视角中$3 \times 3.14$是无比寻常的事,但是在OCaml中,
  你必须强制将整数转换为浮点数,并使用浮点专用的运算符:

\begin{lstlisting}[language=ml]
float_of_int(3) *. 3.14;;
\end{lstlisting}

  \subsubsection*{名义类型与结构类型}
  \indent 类型系统中有一个经典的问题:如何判断两个类型是否相同?
  这个问题通常有两种答案:名义类型(Nominal Typing)和结构类型(Structural Typing)。

  \indent 对于名义类型,两个类型相等当且仅当类型名称相等。这种情况在静态语言中更加常见:

\begin{lstlisting}[language=c++]
struct A {
  int a;
};

struct B {
  int a;
};

// ...
A aa = B(); // error!
\end{lstlisting}

  \indent 上面的C++代码中,尽管两个结构体所包含的字段完全相同,A和B依然是两个不同的类型。
  结构类型与之相反:只要二者在结构上一致,那么两个类型相同。这在一些动态语言中更加常见,例如TypeScript:

\begin{lstlisting}[language=TypeScript]
class A {
  a: string
}

class B {
  a: string
}

const aa: A = new B() // OK!
\end{lstlisting}

  \subsubsection*{类型推断}
  \indent 至此,我们已经看过了大量的代码,以及代码中的类型。无论是写作$int x;$或者是$x: int$。
  这里的$int$都是需要程序员指定的,我们管它叫做类型标记(Type Annotation)。
  总是要不厌其烦地添加这些标记是一件麻烦事。为了减少程序员的工作量,一些类型系统能够自发地识别一个表达式的类型。
  例如$x = 2 * 3.14$,类型系统能够自动推断出$x$的类型是浮点数。
  目前,主流的编程语言大多数都或多或少地支持了类型推断:C++可以使用auto关键字,
  TypeScript可以直接省去类型一部分类型标记(不是全部,函数参数就不行)……

  \indent 等等!为代码添加类型标记,真的是一件很麻烦而且不值得的事情吗?
  至少在工业界中,很多人给出了不同的声音。例如:添加类型标记有的时候可以增加代码的可读性,
  这对于代码后期的维护是至关重要的。从这个角度来看,类型推断其实只是一个可有可无的存在。 \cite{zh}

  \indent 尽管这是一个开放性的问题,我们在后续的内容中仍将假定“类型推断是很重要的一部分”,
  并介绍推断的原理与相关的算法。

  \begin{thebibliography}{5}
    \bibitem{sf} Pierce, B., Amorim, A., Casinghino, C., Gaboardi, M., Greenberg, M., Hriţcu, C., Sjöberg, V. and Yorgey, B., 2018. Software Foundations. pp.1-2.
    \bibitem{typewiki} Zh.wikipedia.org. 2022. 型別系統 - 维基百科,自由的百科全书.[online] Available at: <\url{https://zh.wikipedia.org/wiki/%E9%A1%9E%E5%9E%8B%E7%B3%BB%E7%B5%B1}> [Accessed 26 September 2022].
    \bibitem{gradualwiki} En.wikipedia.org. 2022. Gradual typing - Wikipedia. [online] Available at: <\url{https://en.wikipedia.org/wiki/Gradual_typing}> [Accessed 26 September 2022].
    \bibitem{zh} Zhihu.com. 2021. 如何看待王垠关于Hindley-Milner类型系统的评价? - 知乎. [online] Available at: <\url{https://www.zhihu.com/question/447455754}> [Accessed 26 September 2022].
    \bibitem{typeytb} Youtube.com. 2016. CppCon 2016: Ben Deane “Using Types Effectively". [online] Available at: <\url{https://www.youtube.com/watch?v=ojZbFIQSdl8}> [Accessed 12 October 2022].
  \end{thebibliography}
\end{document}