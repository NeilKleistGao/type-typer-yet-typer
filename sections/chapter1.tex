\documentclass[../main.tex]{subfiles}
\begin{document}
  \section{类型与编程语言}
  \subsection{软件工程师们是如何解决问题的}

  \indent 欢迎来到软件工程的世界!在这里,请允许我假定你已经掌握了一些基本的编程技术(如果还没有的话,为什么不现在就开始学习呢?),
  例如C,C\#,或者Python编程。无论你掌握了哪一门语言,你都已经具备了编写程序的能力,可以开发属于自己的软件了!让我在这里恭喜你!

  \indent 等等!你我其实都心知肚明:这仅仅只是一个开始(如果你确实是一个新晋的程序员的话)。
  你的程序也许并不像你想象中那么完美,比如下面的这个小小的计算程序:

\begin{lstlisting}[language=python]
area = 45
width = input("please input the width of rectangle:")
print(area / int(width))
\end{lstlisting}

  \indent 这是一个非常简单的程序,仅仅只包含3行代码。它可以帮助你计算“当一个矩形的面积为45,宽为width时,矩形的高是多少”。
  但是这样的程序存在着非常严重的缺陷:当用户输入0时,程序将会崩溃(0不能作为除数!):

\begin{lstlisting}
Traceback (most recent call last):
File "script.py", line 5, in <module>
  print(area / int(width))
ZeroDivisionError: division by zero

Exited with error status 1
\end{lstlisting}

  \indent 甚至更糟糕!用户可能根本不会输入一个数字(也许不小心输入了家庭住址)!
  这些情况都会导致我们的程序崩溃,后续的逻辑根本不会执行,从而让用户为此抓狂!
  事实上,不仅仅是我们的程序会遇到这样的问题。很多知名的大型软件(比如Windows系统)的开发者,
  他们也需要不停地思考:怎么样才能让软件更加“好用”呢?

  \indent 这里的“好用”一词实属模糊,不同的人对于“好用”的定义是完全不一样的。
  不过,我们的软件还是应该至少保证两点:按照我们的预期产生输出,并且程序不要崩溃。
  如果程序的输出是错的(比如告诉你$16 \times 55 = 28$),我们的用户压根没法指望着靠这个程序完成什么工作;
  如果程序在运行过程中彻底崩溃了,用户正在进行的工作将付之东流(应该不会有人在Word文档崩溃的时候才发现自己写了两小时的报告忘记保存了吧?)。

  \indent 为此,软件工程师们(以及众多学术界的大牛们)想尽一切办法,提出了一系列的措施,
  目的就是为了减少程序员在开发过程中犯错的次数,从而使用户最终拿到的软件更加可靠。
  这一系列的措施可以大致归纳为三类:\cite{sf}
  \begin{itemize}
    \item 错误终归是由于人引发的。如果不用熬夜甚至通宵写代码的话,代码的正确性就不会依赖于血液中咖啡因的浓度了吧。因此,合理安排项目进度,积极与用户沟通等方法都有助于减少软件中错误的产生。这样的方法我们称之为项目管理;
    \item 如果老的代码是可靠的,那么只需要“复用”以前的工作就好了吧!工程师们通常会设计出一些“模式”,通过使用这些模式来组织代码,后人就能方便地复用这些成果,从而减少新产生的错误的数量。
    \item 会产生错误就是由于程序员写代码的时候太懒散了!多给他们立一些规矩吧!比如“这个地方不允许这样写”,“这个地方你必须输出这个”……这些规矩最后会被反应到编程语言、测试框架或者验证程序上。
  \end{itemize}

  \indent 对于这些措施,我们没有那么多的精力逐个阐述。我们将把注意力放在最后一类:编程语言本身。
  更进一步地,我们将主要讨论编程语言中的类型系统。

  \subsection{编程语言中的类型}
  TODO
  \subsection{不同门派的编程语言}
  TODO
  \subsection{类型系统好处都有啥?}
  TODO

  \begin{thebibliography}{1}
    \bibitem{sf} Pierce, B., Amorim, A., Casinghino, C., Gaboardi, M., Greenberg, M., Hriţcu, C., Sjöberg, V. and Yorgey, B., 2018. Software Foundations. pp.1-2.
  \end{thebibliography}
\end{document}